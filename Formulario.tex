\documentclass[12pt,letterpaper]{article}
\usepackage[utf8]{inputenc}
\usepackage[spanish]{babel}
\usepackage[rmargin=2.5cm,lmargin=2.5cm,tmargin=3cm,bottom=3cm]{geometry}
\usepackage{graphicx}
\usepackage{fancyhdr}
    \pagestyle{fancy}
        \fancyhf{}
        \rhead{\textcolor{black}{Aguirre Vera Jared Emmanuel }}
        \cfoot{\textcolor{black}{\thepage}}

\usepackage[x5names]{xcolor}
\usepackage{caption}
\usepackage[rightcaption]{sidecap}

\usepackage{graphicx}
\usepackage{graphics}
\usepackage{latexsym}
\usepackage{amssymb}
\usepackage{dsfont}
\usepackage{amsmath}
\usepackage{mathrsfs}

\begin{document}
  \lhead{\textcolor{black}{\textsc{Física}}}
\pagecolor{orange}
\color{green}
\begin{center}
    {\textcolor{red}{\textbf{\textsc{\Huge{¿No todo es despejar x?}}}}}
\end{center}
\section*{Física}
\begin{itemize}
    \item[\$] \textbf{\underline{{Velocidad media}}}\newline
    La velocidad media es la relación que existe entre el desplazamiento y el tiempo de un cuerpo y es esencial en el estudio de la mecánica clásica.
    $$V={x1-x0}/{t1-t0}.$$
    \item[\%]\textbf{\underline{{Segunda ley de Newton}}}\newline
    La segunda ley de Newton tambien llamada ley de inercia es una de las tres leyes del movimiento de la mecanica clasica.
     $$F=ma.$$
     \item[\#]\textbf{\underline{{Tensión superficial}}}\newline
     La tensión superficial es un fenómeno en el cual un líquido actúa como una fina lamina.
     $${gamma}=F/d.$$
     \item[\&]\textbf{\underline{{Presión}}}\newline
     La presión que ejerce una fuerza una fuerza está dada por la misma fuerza entre la superficie o área en la que se aplica.
     $$P=F/A.$$
     \item[:] \textbf{\underline{{Energía}}}\newline
     Quizá la fórmula más famosa de la física ideada por Albert Einstein y establece que la energía es igual a la masa.
     $$E=mc^{2}.$$
\end{itemize}
\newpage
  \lhead{\textcolor{black}{\textsc{Química}}}
   \section*{Química}
  \begin{itemize}
      \item[\$] \textbf{\underline{{Molariadad.}}}\newline
     La molaridad se define como el número de moles de un soluto entre el volumen de la disolución.
      $$M = nso / Vsn.$$
      \item[\#] \textbf{\underline{{viscosidad.}}}\newline
      viscosidad cinemática se puede definir como la resistencia de un fluido a moverse como un fluido debe hacerlo.
      $$v={mu}/{rho}.$$
        \item[\%] \textbf{\underline{{Densidad.}}}\newline
        La densidad se refiera a la cantidad de masa en un espacio determinado, es decir es una relación entre la más y el volumen.
        $$P=m/V.$$
        \item[\&] \textbf{\underline{Numero de Moles.}}\newline
        %pues, ¿qué puedo decir? Xd, creo el nombre e bastante explicito.
        El número de moles es una relación entre la masa y el peso de la molécula es como resultado de esto la definición del número de moles.
        $$Nm=m/pm.$$
        \item[:] \textbf{\underline{{Concentración    Porcentual.}}}\newline
        La concentración porcentual, evaluada en masa/volumen, expresa la masa en gramos de soluto en cada 100cm cúbicos de disolución.  
        $$C=m/v.$$
  \end{itemize}
   \newpage
    \lhead{\textcolor{black}{\textsc{Cálculo}}}
    \section*{Cálculo}
    \begin{itemize}
        \item[\$] \textbf{\underline{{Ley de tricotomía.}}}\newline
        Es un axioma de campo y nos habla de que para cualquier número a en los Reales solo puede ser alguno de los tres casos.
        $$(a) {a} = 0,
          (b) {a} pertenece a R+,
          (c) {-a} pertenece a R+.$$
        \item[\#] \textbf{\underline{{Transitividad.}}}\newline
        La transitividad se refiera a una cualidad respecto a al menos dos desigualdades que comparten algún valor y compara los otros dos en función del valor común respecto a la desigualdad con este.
        $$Si, a < b y b < c, entonces, a < c.$$
        \item[\%] \textbf{\underline{{Desigualdad del Triíangulo.}}}\newline
        Esta es una propiedad del valor absuloto y nos dice que la suma del valor absoluto es mayor o igual a la suma de los valores absolutos.
        $$|a| + |b| = |a| + |b|.$$
        \item[\&] \textbf{\underline{Propiedad Arquimediana.}}\newline
        Establece que entre dos números reales cueles quiera existe algún natural tal que crea una relación de desigualdad.
       \begin{center}
        Sea y en R y x mayor que 0, existe n en N tal que.
        $$nx > y.$$
       \end{center} 
        \item[:] \textbf{\underline{{Teorema Bolzano-Weinstass.}}}\newline
        El teorema dice que para toda sucesión acotada tiene al menos una subsucesión convergente.
        $$sea\{An\} Existe \{Ank\} convergente.$$
    \end{itemize}
    \newpage
    \lhead{\textcolor{black}{\textsc{Geometría.}}}
    \section*{Geometría.}
    \begin{itemize}
        \item[\$] \textbf{\underline{{Perímetro.}}}\newline
        El perímetro es igual a la suma de todos los lados de una figura excepto en las que tienen sectores circulares.
        $$P=l1+l2+ln.$$
        \item[\%] \textbf{\underline{{Volumen.}}}\newline
        El volumen de casi cualquier figura es producto del área de la base por la altura de la misma, (solo para figuras cilíndricas).
        $$V=Ah$$
        \item[\&] \textbf{\underline{Teorema de Pitágoras.}}\newline
        El teorema de Pitágoras es una ecuación que nos sirve para calcular la hipotenusa de un triángulo rectángulo a partir de la suma del cuadrado de sus catetos.
        $$h^{2}=Co^{2}+Ca^{2}$$
        \item[\#] \textbf{\underline{{Área para un círculo.}}}\newline
         El área de un círculo es muy particular porque el circulo es una figura muy particular y para obtenerla usamos uno de los símbolos más famosos del mundo {Pi}.
         $$A={Pi}(r{2})$$
        \item[:] \textbf{\underline{{Teorema de Thales.}}}\newline
        Si dos rectas cualesquiera se cortan por varias rectas paralelas, los segmentos determinados en una de las rectas son proporcionales a los segmentos correspondientes en la otra estableciendo una relacion entre segmentos.
       $$A=a,y,E=e$$
    \end{itemize}
    \newpage
    \lhead{\textcolor{black}{\textsc{Matemáticas.}}}
    \section*{Matemáticas.}
    \begin{itemize}
        \item[:] \textbf{\underline{{Conmutatividad.}}}\newline
        Se dice que una operación (suma o producto) es conmutativa si al modificar el orden de sus elementos el resultado de la operación es el mismo.
        $$a+b=b+a,ab=ba.$$
        \item[\#] \textbf{\underline{{Asociatividad.}}}\newline
        Para cuales quiera dos números reales se cumple que la suma y el producto son asociativos es decir se pueden resolver en el orden que sea.
        $$(a+b)+c=a+(b+c), ((a)(b))(c)=(a)((b)(c)).$$
        \item[\&] \textbf{\underline{Inversos.}}\newline
        Para cualquier número real a, distinto de 0, existe un
        número real b tal que.
        $$ab=1.$$
        Para cualquier número real a existe un número real b tal que.
        $$a+b=0.$$
        \item[\%] \textbf{\underline{{Neutros.}}}\newline
        Existe un número real, que denotaremos por
        0, tal que.
        $$a+0=a.$$
         Existe un número real, distinto del
         0, que denotaremos por 1, tal que
        $$a(1)=a.$$
        \item[\%] \textbf{\underline{{Distributividad.}}}\newline
        La distributividad nos dice que el producto de una suma es la suma de los productos.
        $$c(a+b)=((a)(c))+((c)(b)).$$
    \end{itemize}
\end{document}
